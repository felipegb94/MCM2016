\documentclass[pre,12pt]{revtex4-1}
\usepackage{epsfig,graphics,amssymb,amsmath,subeqnarray,setspace,graphicx,amsthm,subfigure, mathrsfs,colortbl,color,bm,fancyhdr}

% Header on each page, with team number and page numbering
\pagestyle{fancy}
\cfoot{}
\renewcommand{\headrulewidth}{0.4pt}
\renewcommand{\footrulewidth}{0.0pt}


\newcommand{\tcb}{\textcolor{blue}}

% Double spacing is 2, 1.5 spacing is 1.5, ...
\def\baselinestretch{1.5}

\begin{document}
\begin{center}
\Large Plans to Locate Missing Flight OA815 Appear Promising


\large Oceanic Airlines

\date{\today}
\end{center}
\normalsize

We are currently involved in an ongoing process of locating an airplane that has recently been downed overseas. 
Unfortunately, we currently have very limited information about the details of flight OA815, as our investigation process is still in its infancy. With the power invested in us, we will set out to find this missing plane with hope of finding survivors, and also retrieving the information contained within the black box. It is important to locate the fuselage of the plane in order to recover the black boxes. Using the recorded flight data information, we can determine the cause of the incident and future incidents can be prevented.

Finding this plane will not be an easy task. The only information we have available to us right now is the planned flight path, and the last documented location in which we had communication with the flight OA815. Unfortunately, the pings sent from the flight data recorder have not been strong enough to detect so far, and we do not expect to receive a ping.

We have developed a practical process in order to help us locate flight  OA815. Our process in trying to locate the remains of the plane involves locating the floating debris. We can compute a possible location of where the debris originated by reversing where the debris had been taken due to surface-level ocean currents. We will, in a sense, retrace the steps of the floating debris.
We will note when and where we spot each debris formation, and create a collection of possible points of where the mass of debris originally diverged from. By averaging these points, it will ideally tell us a very accurate location of where the plane hit the water. From where we think the plane crashed, we will then deploy a group of submarines with an active sonar detecting system with hopes of being able to detect the downed plane. 

We will able to identify a probable location accurately with this method, but not an exact one. This process is will help us narrow down a search grid to tell us where to focus most of our efforts.

Our initial locating process will start above the ocean utilizing various search airplanes. We can and will use all available resources to locate debris as we are not looking for anything else at the moment that requires any more technology than just being able to spot debris. 

We would like to send our sincerest condolences to the family members affected by the tragedy of flight  OA815, and we would like make it very clear that our priority is to find the plane as quickly, as safely, and efficient as possible.	

\end{document}

