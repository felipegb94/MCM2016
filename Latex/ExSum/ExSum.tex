\documentclass[notitlepage]{report}
\usepackage{float}
\usepackage{epsfig,graphics,amssymb,amsmath,subeqnarray,setspace,graphicx,amsthm,subfigure, mathrsfs,colortbl,color,bm,fancyhdr,wrapfig,tikz}

% Header on each page, with team number and page numbering
\pagestyle{fancy}
\lhead{Team \# 52821} 
\rhead{page \thepage \ of \pageref{LastPage}}
\cfoot{}
\renewcommand{\headrulewidth}{0.4pt}
\renewcommand{\footrulewidth}{0.0pt}

% Make life easier: define some shortcuts!
\def\b{\bm}
\def\e{\epsilon}
\def\ep{\varepsilon}
\def\u{\underline}
\def\c{\centerline}
\def\n{\noindent}
\def\h{\hangindent}


\newcommand{\tcb}{\textcolor{blue}}

% Double spacing is 2, 1.5 spacing is 1.5, ...
\def\baselinestretch{1.0}

% For tables:
\definecolor{Gray}{gray}{0.9}
\newcolumntype{C}{c<{\kern\tabcolsep}@{}}

\title{Executive Summary}
\author{Team \# 52821}
\date{\today}
\begin{document}

\maketitle
\thispagestyle{empty}

On February 10, 2009 USA satellite Iridium-33 collided with Russian satellite Kosmos-2251 in the most severe space collision to date. This costly event is one example of many regularly occurring collisions that act as the root of thousands if not millions of debris orbiting our planet. While the collisions usually occur among smaller pieces of debris they can have serious ramifications. \\


Left unchecked, these collisions can not only cause catastrophic damage to significant assets, but they also create more debris, increasing the probability for future collisions. Overtime, this model becomes unstable and future space travel can become impossible. These debris collisions can cost billions of dollars to spacefaring parties and can provide fatal risk to any manned mission. Given the current state and prospective future of space travel, the importance of safe and affordable space exploration has never been greater. \\

When analyzing space debris reduction methods short-term and long-term effects must be considered. Further, given the state of the economy, solutions must be commercially viable. While there are many solutions, the following showed initial promise and were considered for evaluation:
\begin{itemize}
	\item Debris Capture and Removal via:
	\begin{itemize}
		\item Robotic Arms
		\item Tethers
		\item Net Capturing
	\end{itemize}

	\item Laser Based Ablation
	\begin{itemize}
		\item Ground Based
		\item Space Based
	\end{itemize}
	\item No Action
\end{itemize}


All Debris Capture and Removal solutions were economically unviable. They require a great deal of energy to maintain the high velocities needed to capture targets. These solutions were no longer considered due to the unrealistic cost. \\

\newpage
%%%%%%%%%%%%%%%%%%%%%%%%%

The No Action solution was explored first. While shielding and avoidance maneuvers will always be used to increase safety, they will not be enough to guarantee the safety for long-term missions. A significant finding of our model predicts a dramatic increase of catastrophic collisions within our lifetime. This is corroborated by numerous other models computed from respected groups such as the National Aeronautics and Space Administration (NASA). Short and long term solutions are both high priority initiatives. \\


Testing various reduction scenarios indicated specific levels of reduction were optimal. Another significant finding indicated the reduction at a high level for five years gave the same long term reduction results as small levels for decades. This demonstrates that a higher effort up front can drastically improve the current situation while also providing the same long term debris security. This was the preferred solution. \\


Numerous Laser Based Ablation systems were studied to find the optimal configuration for the preferred solution. A space based solution was found to be the preferred solution as well as the most financially feasible. Our model predicts that a cost of \$400 million over the course of five years will significantly improve the state of space travel for at least the next century. This is the recommended solution. \\


The cost of this recommendation may seem substantial, but over five years the cost is only \$80 million per year. This also provides at least a century of space debris security. The averaged cost roughly becomes just \$4 million per year, but is insignificant to possible savings. All space fairing parties will benefit from the proposed solution. The ability for all parties to peacefully and safely explore space is the crux for future space innovation and discovery. These are the steps that can be taken to ensure that this discovery can continue to happen long into the future. \\


\end{document}